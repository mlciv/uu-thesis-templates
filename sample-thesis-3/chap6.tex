%%% -*-LaTeX-*-

\chapter{Sample theorem-like environments}

The standard \LaTeX{} \verb=\newtheorem= declaration produces numbered
theorems.  However, the \verb=amsthm= package offers a
\verb=\newtheorem*= declaration to suppress numbering.

By default, the theorem-like declarations number their output blocks
consecutively throughout the document.  However, with an optional
bracketed third argument, you can number blocks within specified
sectional units.  We give several examples in this chapter.

%%% --------------------------------------------------------------------

\section{Default-numbered samples}

The \LaTeX{} input for this section begins:
%
\begin{verbatim}
\newtheorem  {guess}      {Conjecture}
\newtheorem* {prediction} {Unnumbered prediction}
\begin{guess}
    The Boston Red Sox will win next year's championship.
\end{guess}
\begin{guess}
    The Pittsburgh Pirates will \emph{not} win next year's championship.
\end{guess}
\begin{prediction}
    The New York Yankees will be sold to a Denver mining conglomerate.
\end{prediction}
\begin{prediction}
    Baseball will overtake football in television broadcast revenues.
\end{prediction}
\end{verbatim}

%
\newtheorem  {guess}      {Conjecture}

\newtheorem* {prediction} {Unnumbered prediction}

\begin{guess}
    The Boston Red Sox will win next year's championship.
\end{guess}

\begin{guess}
    The Pittsburgh Pirates will \emph{not} win next year's championship.
\end{guess}

\begin{prediction}
    The New York Yankees will be sold to a Denver mining conglomerate.
\end{prediction}

\begin{prediction}
    Baseball will overtake football in television broadcast revenues.
\end{prediction}

%%% --------------------------------------------------------------------

\section{Chapter-numbered samples}

The \LaTeX{} input for this section begins:
%
\begin{verbatim}
\newtheorem  {guess-2} {Conjecture} [chapter]
\begin{guess-2}
    The Boston Red Sox will win next year's championship.
\end{guess-2}
\begin{guess-2}
    The Pittsburgh Pirates will \emph{not} win next year's championship.
\end{guess-2}
\end{verbatim}

%
\newtheorem {guess-2} {Conjecture} [chapter]

\begin{guess-2}
    The Boston Red Sox will win next year's championship.
\end{guess-2}

\begin{guess-2}
    The Pittsburgh Pirates will \emph{not} win next year's championship.
\end{guess-2}

%%% --------------------------------------------------------------------

\section{Section-numbered samples}

The \LaTeX{} input for this section begins:
%
\begin{verbatim}
\newtheorem {guess-3} {Conjecture} [section]
\begin{guess-3}
    The Boston Red Sox will win next year's championship.
\end{guess-3}
\begin{guess-3}
    The Pittsburgh Pirates will \emph{not} win next year's championship.
\end{guess-3}
\end{verbatim}

%
\newtheorem {guess-3} {Conjecture} [section]

\begin{guess-3}
    The Boston Red Sox will win next year's championship.
\end{guess-3}

\begin{guess-3}
    The Pittsburgh Pirates will \emph{not} win next year's championship.
\end{guess-3}

%%% --------------------------------------------------------------------

\subsection{Subsection-numbered samples}

The \LaTeX{} input for this section begins:
%
\begin{verbatim}
\newtheorem {guess-4} {Conjecture} [subsection]
\newtheorem {hunch-4} {Hunch}      [subsection]
\begin{guess-4}
    The Boston Red Sox will win next year's championship.
\end{guess-4}
\begin{guess-4}
    The Pittsburgh Pirates will \emph{not} win next year's championship.
\end{guess-4}
\begin{hunch-4}
    The Salt Lake Bees will not make the major leagues.
\end{hunch-4}
\end{verbatim}

%
\newtheorem {guess-4} {Conjecture} [subsection]
\newtheorem {hunch-4} {Hunch}      [subsection]

\begin{guess-4}
    The Boston Red Sox will win next year's championship.
\end{guess-4}

\begin{guess-4}
    The Pittsburgh Pirates will \emph{not} win next year's championship.
\end{guess-4}

\begin{hunch-4}
    The Salt Lake Bees will not make the major leagues.
\end{hunch-4}

%%% --------------------------------------------------------------------

\subsection{More subsection-numbered samples}

The \LaTeX{} input for this section begins:
%
\begin{verbatim}
\newtheorem {guess-5} {Conjecture} [subsection]
\newtheorem {hunch-5} {Hunch}      [guess-5]
\begin{guess-5}
    The Boston Red Sox will win next year's championship.
\end{guess-5}
\begin{guess-5}
    The Pittsburgh Pirates will \emph{not} win next year's championship.
\end{guess-5}
\begin{hunch-5}
    The Salt Lake Bees will not make the major leagues.
\end{hunch-5}
\end{verbatim}

%
\newtheorem {guess-5} {Conjecture} [subsection]
\newtheorem {hunch-5} {Hunch}      [guess-5]

\begin{guess-5}
    The Boston Red Sox will win next year's championship.
\end{guess-5}

\begin{guess-5}
    The Pittsburgh Pirates will \emph{not} win next year's championship.
\end{guess-5}

\begin{hunch-5}
    The Salt Lake Bees will not make the major leagues.
\end{hunch-5}
